\documentclass{article}

\begin{document}



\centerline{Responses to critiques}

\centerline{Group 26}
\vspace{2pc}

\section{Group 32}
Addressing the lack of understanding on the biological aspects of the project. We address this in the intro and abstract for our paper. However, to simplify things, cell segmentation is an important problem because it has a wide array of biological applications to finding the attributes like nodes, branches, branch lengths, loops etc for ER networks. We address these in the abstract of our paper. 

\section{Group 21}
We address the specific steps taken to do the preprocessing in the paper and also in the demo, however we understand that this may not have been made as clear during the slides. 
To address why we selected the U-Net over other alternatives:
We address this in the previous work section of the paper. U-net is faster for testing on large sets of data once the model is trained for a specific purpose (in our case, identifying ER-network). We attempted to use the existing alternative on ImageJ's inbuilt WEKA segmentation. However, to get the accuracy that can be achieved with U-net, we needed to manually train several images which took a lot of time. 
We will attempt to be less verbose on the slides themselves on future presentations.

\section{Group 27}

We will try to be clearer about the background section on the paper, but the final conclusion that you made about our ethos was correct. Zoom was a bit awkward to use for transitioning since it was new, and hopefully we will transition into being better with that. 


"ADDRESS METRICS: There is something called a dice coefficient. But I need to look that up: Anamika"





\end{document}